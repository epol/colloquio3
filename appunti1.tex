\documentclass[a4paper,10pt]{article}

\usepackage{amsmath}
\usepackage{amssymb}
\usepackage{amsthm}
\usepackage{xfrac}
\usepackage[all]{xy}
\usepackage{mathtools}
\usepackage{graphicx}
\usepackage{fullpage}
\usepackage{hyperref}
\usepackage[utf8x]{inputenc}
\usepackage[italian]{babel}

\usepackage{ulem}

\setlength{\parindent}{0in}

\theoremstyle{plain}
\newtheorem{myteo}{Teorema}[section]
\newtheorem{mylem}{Lemma}[section]
\newtheorem{mypro}{Proposizione}[section]
\newtheorem{mycor}{Corollario}[section]
\newtheorem*{myteo*}{Teorema}
\newtheorem*{mylem*}{Lemma}
\newtheorem*{mypro*}{Proposizione}
\newtheorem*{mycor*}{Corollario}

\theoremstyle{definition}
\newtheorem{mydef}{Definizione}[section]
\newtheorem{myes}{Esempio}[section]
\newtheorem{myex}{Esercizio}[section]
\newtheorem*{mydef*}{Definizione}
\newtheorem*{myes*}{Esempio}
\newtheorem*{myex*}{Esercizio}

\theoremstyle{remark}
\newtheorem{mynot}{Nota}[section]
\newtheorem{myoss}{Osservazione}[section]
\newtheorem*{mynot*}{Nota}
\newtheorem*{myoss*}{Osservazione}

\newcommand{\obar}[1]{\overline{#1}}
\newcommand{\ubar}[1]{\underline{#1}}

\newcommand{\set}[1]{\left\{#1\right\}}
\newcommand{\pa}[1]{\left(#1\right)}
\newcommand{\ang}[1]{\left<#1\right>}
\newcommand{\bra}[1]{\left[#1\right]}
\newcommand{\abs}[1]{\left|#1\right|}
\newcommand{\norm}[1]{\left\|#1\right\|}

\newcommand{\pfrac}[2]{\pa{\frac{#1}{#2}}}
\newcommand{\bfrac}[2]{\bra{\frac{#1}{#2}}}
\newcommand{\psfrac}[2]{\pa{\sfrac{#1}{#2}}}
\newcommand{\bsfrac}[2]{\bra{\sfrac{#1}{#2}}}

\newcommand{\der}[2]{\frac{\partial #1}{\partial #2}}
\newcommand{\pder}[2]{\pfrac{\partial #1}{\partial #2}}
\newcommand{\sder}[2]{\sfrac{\partial #1}{\partial #2}}
\newcommand{\psder}[2]{\psfrac{\partial #1}{\partial #2}}

\newcommand{\intl}{\int \limits}

\DeclareMathOperator{\de}{d}
\DeclareMathOperator{\id}{Id}


\title{Colloquio 3 - appunti1}
\author{Enrico Polesel}
\date{\today}

\begin{document}
\maketitle

Ipotesi: $(M,d)$ spazio metrico

Ipotesi: $\Omega \subseteq M$

\begin{mydef}[Distanza di un punto da un insieme]
\[  d_A (x) := \inf _{y \in A} d(x,y) \] 
\end{mydef}

\textbf{Idea:} forme $\leftrightarrow$ sottoinsiemi chiusi
$\leftrightarrow$ funzioni $d_A$

\begin{mydef}[Spazio di forme]
  \[ C_d(\Omega) = \set{d_A : A \subseteq \Omega} \]
\end{mydef}

\begin{mypro}[Proprietà della distanza punto-insieme]
  Vale:
  \begin{itemize}
  \item La funzione $x \rightarrow d_A(x)$ è 1-Lipschitz (e quindi
    $C^1$)
  \item $d_{A\cup B} = \min\set{d_A,d_B}$
  \item $\set{x:d_A(x) = 0} = \bar A$
  \item $d_A \equiv 0 \Rightarrow $ è denso
  \item $d_{\bar A} = d_A$
  \item $A\subseteq B \Rightarrow d_A \ge d_B$
  \item $\bar A \subseteq \bar B \Leftrightarrow d_A \ge d_B$
  \end{itemize}
\end{mypro}
\begin{proof} [Dimostrazione di $\set{x:d_A(x) = 0} = \bar A$]
  \`E ovvio che $x\in A \Rightarrow d_A(x) = 0$

  Sia $x\not\in \bar A$, allora esiste $\varepsilon>0$ tale che
  $B(x,\varepsilon) \cap A = \emptyset$. Allora $d_A(x) \ge
  \varepsilon$.
\end{proof}

\begin{proof}[Dimostrazione di $\bar A \subseteq \bar B
  \Leftrightarrow d_A \ge d_B$]
  Essendo $d_B \ge 0$ si ha che $d_A = 0 \Rightarrow d_B = 0$. Ma
  questo significa che 
  \[ x \in \bar A \Rightarrow x \in \bar B \]

  Se $\bar A \subseteq \bar B$ assumiamo wlog $A = \bar A, B = \bar B$
  (le funzioni distanza non cambiano passando alla chiusura). Allora
  la disuguaglianza è verificata perché stiamo facendo l'$\inf$ su due
  insiemi contenuti.  
\end{proof}

\begin{myoss}
  Usare le funzioni $d_A$ per classificare gli insiemi significa
  vederli a meno di chiusure, quindi quozientiamo per
  \[ A \sim B \Leftrightarrow \bar A = \bar B \] 
\end{myoss}


\begin{mydef}[Distanza di Hausdorff]
  Dati due insiemi $A,B \subseteq \Omega$ definiamo
  \[ d_H (A,C) = \norm{ d_A - d_C } _\infty = \sup _{x\in \Omega}
  \set {d_A(x) - d_C(x)}\]
\end{mydef}

Notazione: definiamo (con un abuso di linguaggio):
\[ A + \bar B_\varepsilon := \set{ x \in \Omega : \exists y \in A :
  d(x,y) \le \varepsilon }\]

\begin{mypro}
  Se $A, C$ sono \sout{chiusi} compatti\footnote{Uso la compattezza
    per estrarre sottosucessioni convergenti dai punti dati
    dagl'$\inf$} le seguenti definizioni sono equivalenti
  \begin{itemize}
  \item $ d_H (A,C) = \norm{ d_A - d_C } _\infty = \sup _{x\in \Omega}
    \set {d_A(x) - d_C(x)}$
  \item $d_H(A,C) = \max \set{\sup _{x\in A} d_C (x) , \sup _{x\in C}
      d_A (x) }$
  \item $d_H (A,C) = \inf \set{\delta : A \subseteq C + \bar B
    _\delta \wedge C \subseteq A + \bar B_\delta} = \max \set{\inf
    \set{\delta : A \subseteq C + \bar B _\delta }, \set{\delta : C
      \subseteq A + \bar B _\delta }}  $
  \end{itemize}
\end{mypro}

\begin{mylem}
  Se $A,C$ sono compatti allora gli $\inf$ e i $\sup$ precedenti sono
  dei $\min$ e $\max$
\end{mylem}
\begin{proof}
  Consideriamo la definizione di $d_A$, se $d_A(x) = \delta$ allora esiste
  $\pa{a_n}_{n\in \mathbb{N}}$ con $a_n \in A$ tale che
  \[ d(a_n,x) < \delta + \frac{1}{n} \] Ma, per compattezza, estraiamo
  una sottosuccessione $\pa{a_{n_k}}_{k\in \mathbb{N}}$ che converge
  ad un punto in $A$ che chiamiamo $\bar a$.  Allora, scelto $n_k$
  tale che $d(a_{n_k},\bar a) < \frac{1}{n}$ e che $n_k > n$ si ha
  \[ d(x,\bar a) \le d\pa{x,a_{n_k}} + d\pa{a_{n_k}, \bar a} \le
  \frac{1}{n_k} + \frac{1}{n} \le \frac{2}{n} \]
  Per $n \to \infty$ si ha $d(x,\bar a) \le \delta$, ma $\bar a \in A$
  dà la disuguaglianza opposta da cui $\bar a$ è il punto cercato.

\end{proof}


\newpage

Cose possibili da fare



\newpage

Trovare le ipotesi giuste per:
\begin{itemize}
\item ???? $d_A$ è Frechet differenziabile q.o. e $\abs{\nabla d_A}
  \le 1$
\end{itemize}


\newpage
Ipotesi: $\mathbb{R}^N$

I chiusi e limitati sono compatti, quindi riesco a trasformare gli
$\inf$ in $\min$ mettendomi in una palla $B(y,d+\varepsilon)$ quando
lavoro con chiusi




\end{document}

