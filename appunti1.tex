\documentclass[a4paper,10pt]{article}

\usepackage{amsmath}
\usepackage{amssymb}
\usepackage{amsthm}
\usepackage{xfrac}
\usepackage[all]{xy}
\usepackage{mathtools}
\usepackage{graphicx}
\usepackage{fullpage}
\usepackage{hyperref}
\usepackage[utf8x]{inputenc}
\usepackage[italian]{babel}

\usepackage{ulem}

\setlength{\parindent}{0in}

\theoremstyle{plain}
\newtheorem{myteo}{Teorema}[section]
\newtheorem{mylem}{Lemma}[section]
\newtheorem{mypro}{Proposizione}[section]
\newtheorem{mycor}{Corollario}[section]
\newtheorem*{myteo*}{Teorema}
\newtheorem*{mylem*}{Lemma}
\newtheorem*{mypro*}{Proposizione}
\newtheorem*{mycor*}{Corollario}

\theoremstyle{definition}
\newtheorem{mydef}{Definizione}[section]
\newtheorem{myes}{Esempio}[section]
\newtheorem{myex}{Esercizio}[section]
\newtheorem*{mydef*}{Definizione}
\newtheorem*{myes*}{Esempio}
\newtheorem*{myex*}{Esercizio}

\theoremstyle{remark}
\newtheorem{mynot}{Nota}[section]
\newtheorem{myoss}{Osservazione}[section]
\newtheorem*{mynot*}{Nota}
\newtheorem*{myoss*}{Osservazione}

\newcommand{\obar}[1]{\overline{#1}}
\newcommand{\ubar}[1]{\underline{#1}}

\newcommand{\set}[1]{\left\{#1\right\}}
\newcommand{\pa}[1]{\left(#1\right)}
\newcommand{\ang}[1]{\left<#1\right>}
\newcommand{\bra}[1]{\left[#1\right]}
\newcommand{\abs}[1]{\left|#1\right|}
\newcommand{\norm}[1]{\left\|#1\right\|}

\newcommand{\pfrac}[2]{\pa{\frac{#1}{#2}}}
\newcommand{\bfrac}[2]{\bra{\frac{#1}{#2}}}
\newcommand{\psfrac}[2]{\pa{\sfrac{#1}{#2}}}
\newcommand{\bsfrac}[2]{\bra{\sfrac{#1}{#2}}}

\newcommand{\der}[2]{\frac{\partial #1}{\partial #2}}
\newcommand{\pder}[2]{\pfrac{\partial #1}{\partial #2}}
\newcommand{\sder}[2]{\sfrac{\partial #1}{\partial #2}}
\newcommand{\psder}[2]{\psfrac{\partial #1}{\partial #2}}

\newcommand{\intl}{\int \limits}

\DeclareMathOperator{\de}{d}
\DeclareMathOperator{\id}{Id}


\title{Colloquio 3 - appunti1}
\author{Enrico Polesel}
\date{\today}

\begin{document}
\maketitle

Ipotesi: $(M,d)$ spazio metrico

Ipotesi: $\Omega \subseteq M$

\begin{mydef}[Distanza di un punto da un insieme]
\[  d_A (x) := \inf _{y \in A} d(x,y) \] 
\end{mydef}

\textbf{Idea:} forme $\leftrightarrow$ sottoinsiemi chiusi
$\leftrightarrow$ funzioni $d_A$

\begin{mydef}[Spazio di forme]
  \[ C_d(\Omega) = \set{d_A : A \subseteq \Omega} \]
\end{mydef}

\begin{mypro}[Proprietà della distanza punto-insieme]
  Vale:
  \begin{itemize}
  \item La funzione $x \rightarrow d_A(x)$ è 1-Lipschitz (e quindi
    $C^1$)
  \item $d_{A\cup B} = \min\set{d_A,d_B}$
  \item $\set{x:d_A(x) = 0} = \bar A$
  \item $d_A \equiv 0 \Rightarrow $ è denso
  \item $d_{\bar A} = d_A$
  \item $A\subseteq B \Rightarrow d_A \ge d_B$
  \item $\bar A \subseteq \bar B \Leftrightarrow d_A \ge d_B$
  \end{itemize}
\end{mypro}
\begin{proof} [Dimostrazione di $\set{x:d_A(x) = 0} = \bar A$]
  \`E ovvio che $x\in \bar A \Rightarrow d_A(x) = 0$ perché 
  \[ \forall \varepsilon > 0\; \exists y \in a \cap
  B(x,\varepsilon) \Rightarrow d_A(x) \le d(x,y) < \varepsilon \]

  Sia $x\not\in \bar A$, allora esiste $\varepsilon>0$ tale che
  $B(x,\varepsilon) \cap A = \emptyset$. Allora $d_A(x) \ge
  \varepsilon$.
\end{proof}

\begin{proof}[Dimostrazione di $\bar A \subseteq \bar B
  \Leftrightarrow d_A \ge d_B$]
  Essendo $d_B \ge 0$ si ha che $d_A = 0 \Rightarrow d_B = 0$. Ma
  questo significa che 
  \[ x \in \bar A \Rightarrow x \in \bar B \]

  Se $\bar A \subseteq \bar B$ assumiamo wlog $A = \bar A, B = \bar B$
  (le funzioni distanza non cambiano passando alla chiusura). Allora
  la disuguaglianza è verificata perché stiamo facendo l'$\inf$ su due
  insiemi contenuti.  
\end{proof}


\begin{mylem}
  Se $A$ è compatto allora l'$\inf$ della definizione di $d_A$ è in
  realtà un $\min$, cioè:
  \[ \forall x \in \Omega \; \exists y \in A \ :\ d(y,x) = d_A(x) \]
\end{mylem}
\begin{proof}
  Se $d_A(x) = \delta$ allora esiste $\pa{a_n}_{n\in \mathbb{N}}$ con
  $a_n \in A$ tale che
  \[ d(a_n,x) < \delta + \frac{1}{n} \] 
  Ma, per compattezza, estraiamo una sottosuccessione
  $\pa{a_{n_k}}_{k\in \mathbb{N}}$ che converge ad un punto in $A$ che
  chiamiamo $\bar a$.  Allora, scelto $n_k$ tale che $d(a_{n_k},\bar
  a) < \frac{1}{n}$ e che $n_k > n$ si ha
  \[ d(x,\bar a) \le d\pa{x,a_{n_k}} + d\pa{a_{n_k}, \bar a} \le
  \frac{1}{n_k} + \frac{1}{n} \le \frac{2}{n} \]
  Per $n \to \infty$ si ha $d(x,\bar a) \le \delta$, ma $\bar a \in A$
  dà la disuguaglianza opposta da cui $\bar a$ è il punto cercato.
\end{proof}

\begin{mynot}
  Se $A$ non è compatto ma è un chiuso allora, in generale, il lemma
  precedente non è verificato. Costriuisco un controesempio in $l^2$
  
  Sia, per $n\in \mathbb{N}$ $a_n$ la successione con $1+\frac{1}{n}$
  all'$n$-esimo posto e $0$ altrove. Sia $A = \set{a_n : n\in
    \mathbb{N}}$. Si pu\`o dimostrare che $A$ \`e chiuso utilizzando
  la propriet\`a $\bar A = \set {d_A = 0}$.

  Considero l'origine:
  \[ d_A(O) = \inf _{n\in \mathbb{N}} \norm{a_n}_2 = \inf _{n\in
    \mathbb{N}} \pa{1+\frac{1}{n}}^2 = 1 \]
  Però $\forall n\in \mathbb{N}\; d(O,a_n) > 1$ e quindi non si
  raggiunge il minimo per nessun punto.  
\end{mynot}

\begin{mynot}
  In uno spazio vettoriale di dimensione finita il lemma vale
  chiedenso solo che $A$ sia chiuso, infatti tutti i punti $a_n$ della
  successione data dall'$\inf$ stanno in 
  \[ a_n \in A \cap \bar B(x,2*\delta) \]
  Che è chiusa perché intersezione di chiusi, è limitata e quindi
  anche compatta. Da qui si conclude in modo analogo al lemma.
\end{mynot}


\begin{myoss}
  Usare le funzioni $d_A$ per classificare gli insiemi significa
  vederli a meno di chiusure, quindi quozientiamo per
  \[ A \sim B \Leftrightarrow \bar A = \bar B \] 
\end{myoss}


\begin{mydef}[Distanza di Hausdorff]
  Dati due insiemi $A,B \subseteq \Omega$ definiamo
  \[ d_H (A,C) = \norm{ d_A - d_C } _\infty = \sup _{x\in \Omega}
  \set {d_A(x) - d_C(x)}\]
\end{mydef}

Notazione: definiamo (con un abuso di linguaggio):
\[ A + \bar B_\varepsilon := \set{ x \in \Omega : \exists y \in A :
  d(x,y) \le \varepsilon }\]

\begin{mypro}
  Se $A, C$ sono \sout{compatti} \sout{chiusi} le seguenti definizioni
  sono equivalenti
  \begin{enumerate}
  \item $ d_H (A,C) = \norm{ d_A - d_C } _\infty = \sup _{x\in \Omega}
    \set {d_A(x) - d_C(x)}$
  \item $d_H(A,C) = \max \set{\sup _{x\in A} d_C (x) , \sup _{x\in C}
      d_A (x) }$
  \item $d_H (A,C) = \inf \set{\delta : A \subseteq C + \bar B
    _\delta \wedge C \subseteq A + \bar B_\delta} = \max \set{\inf
    \set{\delta : A \subseteq C + \bar B _\delta }, \set{\delta : C
      \subseteq A + \bar B _\delta }}  $
  \end{enumerate}
\end{mypro}

\begin{proof}
  \textbf{$2 \sim 3$} Basta dimostrare che $\sup _{x\in A} d_C(x) =
  \inf \set{ \delta : A \subseteq C + \bar B _{\delta}}$.

  Se $\sup _{x\in A} d_C(x) = \alpha$ allora
  \[ \forall x \in A \; d_C (x) \le \alpha \Rightarrow \pa{\forall
    \varepsilon > 0 \exists y \in C \mid d(x,y) < \alpha +
    \varepsilon} \Rightarrow \pa{\forall \varepsilon > 0 \; A
    \subseteq C + \bar B _{\alpha + \varepsilon}} \Rightarrow \inf
  \set{ \delta : A \subseteq C + \bar B _{\delta}} \le \alpha \]

  D'altra parte se $\varepsilon > 0$ allora $\exists x \in A$ tale che
  $d_C(x) > \alpha - \varepsilon$. Allora
  \[ \alpha - \varepsilon < \inf_{y\in C}\set{d(x,y)} \Rightarrow
  \forall y \in C\; d(x,y) > \alpha - \varepsilon \Rightarrow x\not\in
  C + \bar B_{\alpha - \varepsilon} \Rightarrow \inf \set{ \delta : A
    \subseteq C + \bar B _{\delta}} \ge \alpha - \varepsilon\]
  Per $\varepsilon \rightarrow 0$ si ha quindi l'altra disuguaglianza
  ottenendo la tesi.

  \textbf{$1\sim 2$}
  Sia $\max \set{\sup _{x\in A} d_C (x) , \sup _{x\in C} d_A (x) } =
  \delta$.
  
  Voglio valutare $\sup_{x\in \Omega} \abs{d_A(x)-d_C(x)}$.

  Primo caso: $x\in A$. Allora
  \[ d_A(x) = 0 \Rightarrow \abs{d_A(x)-d_C(x)} = d_C(x) \le \sup
  _{x\in A} d_C (x) \le \delta \]
  Secondo caso: $x\in C$, in modo analogo al caso precedente otteniamo
  $\abs{d_A(x)-d_C(x)} \le \delta$.
  
  Caso generale: sia $x\in \Omega$, supponiamo wlog $d_A(x) \ge d_C(x)$
  (l'altro caso vale per simmetria). Quindi$\abs{d_A(x)-d_C(x)} =
  d_A(x)-d_C(x)$. Scegliendo $y \in C$ tale che $d(x,y) < d_C(x) +
  \varepsilon$ si ha
  \[ d_A(x)-d_C(x) \le d_A(y) + d(x,y) - d_C(x) < d_A(y) + d_C(x) +
  \varepsilon -d_C(x) \]
  Per $\varepsilon \rightarrow 0$ si ha $d_A(x) \le \delta$.

  Dimostriamo ora che il $\sup_{x\in \Omega} \abs{d_A(x)-d_C(x)} \ge
  \delta$. Per farlo supponiamo wlog $\delta \max \set{\sup _{x\in A}
    d_C (x) , \sup _{x\in C} d_A (x) } = \set{\sup _{x\in A} d_C (x)}$
  (nell'altro caso si conclude con un ragionamento analogo).

  Quindi esiste $\pa{x_n}_{n\in\mathbb{N}}$, $x_n\in A$ tale che
  $d_C(x_n) > \delta - \frac{1}{n}$. Ma allora
  \[ \abs{d_A(x_n) - d_C(x_n)} = d_C(x_n) > \delta - \frac{1}{n} \]
  Facendo tendere $n$ all'infinito si ha la disuguaglianza cercata da
  cui la tesi.
\end{proof}













\newpage

Cose possibili da fare:

\begin{itemize}
\item Vedere quando, nelle definizioni di $d_H$ gli $\inf$ e $\sup$ si
  possono trasformare in $\min$ e $\max$.
\end{itemize}



\newpage

Trovare le ipotesi giuste per:
\begin{itemize}
\item ???? $d_A$ è Frechet differenziabile q.o. e $\abs{\nabla d_A}
  \le 1$
\end{itemize}


\newpage
Ipotesi: $\mathbb{R}^N$

I chiusi e limitati sono compatti, quindi riesco a trasformare gli
$\inf$ in $\min$ mettendomi in una palla $B(y,2*d+\varepsilon)$ quando
lavoro con chiusi


\newpage
Cose saltate che potrebbero essere inserite
\begin{itemize}
\item skeleton
  \begin{itemize}
  \item caratterizzazione coi punti dove non esiste il gradiente di
    $d_A$
  \end{itemize}
\end{itemize}


\end{document}

