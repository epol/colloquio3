\documentclass{beamer}
%\usetheme{PaloAlto}
%\usetheme{Berlin}
\usetheme{Ilmenau}
\usecolortheme{seahorse}

%\usepackage[utf8]{inputenc}
%\usepackage{default}
%\usepackage[italian]{babel}

%\usepackage{titleref}
%\usepackage{zref-titleref}

\usepackage{amsmath}
\usepackage{amssymb}
\usepackage{amsthm}
\usepackage{xfrac}
\usepackage[all]{xy}
\usepackage{mathtools}
\usepackage{graphicx}
%\usepackage{fullpage}
\usepackage{hyperref}
\usepackage[utf8x]{inputenc}
\usepackage[italian]{babel}

\usepackage{pdftricks}
\begin{psinputs}
   \usepackage{pstricks}
   \usepackage{multido}
\end{psinputs}

\usepackage{ulem}

\setlength{\parindent}{0in}

\newcounter{counter1}

\theoremstyle{plain}
\newtheorem{myteo}[counter1]{Teorema}
\newtheorem{mylem}[counter1]{Lemma}
\newtheorem{mypro}[counter1]{Proposizione}
\newtheorem{mycor}[counter1]{Corollario}
\newtheorem*{myteo*}{Teorema}
\newtheorem*{mylem*}{Lemma}
\newtheorem*{mypro*}{Proposizione}
\newtheorem*{mycor*}{Corollario}

\theoremstyle{definition}
\newtheorem{mydef}[counter1]{Definizione}
\newtheorem{myes}[counter1]{Esempio}
\newtheorem{myex}[counter1]{Esercizio}
\newtheorem*{mydef*}{Definizione}
\newtheorem*{myes*}{Esempio}
\newtheorem*{myex*}{Esercizio}

\theoremstyle{remark}
\newtheorem{mynot}[counter1]{Nota}
\newtheorem{myoss}[counter1]{Osservazione}
\newtheorem*{mynot*}{Nota}
\newtheorem*{myoss*}{Osservazione}

\newcommand{\obar}[1]{\overline{#1}}
\newcommand{\ubar}[1]{\underline{#1}}

\newcommand{\set}[1]{\left\{#1\right\}}
\newcommand{\pa}[1]{\left(#1\right)}
\newcommand{\ang}[1]{\left<#1\right>}
\newcommand{\bra}[1]{\left[#1\right]}
\newcommand{\abs}[1]{\left|#1\right|}
\newcommand{\norm}[1]{\left\|#1\right\|}

\newcommand{\pfrac}[2]{\pa{\frac{#1}{#2}}}
\newcommand{\bfrac}[2]{\bra{\frac{#1}{#2}}}
\newcommand{\psfrac}[2]{\pa{\sfrac{#1}{#2}}}
\newcommand{\bsfrac}[2]{\bra{\sfrac{#1}{#2}}}

\newcommand{\der}[2]{\frac{\partial #1}{\partial #2}}
\newcommand{\pder}[2]{\pfrac{\partial #1}{\partial #2}}
\newcommand{\sder}[2]{\sfrac{\partial #1}{\partial #2}}
\newcommand{\psder}[2]{\psfrac{\partial #1}{\partial #2}}

\newcommand{\intl}{\int \limits}

\DeclareMathOperator{\de}{d}
\DeclareMathOperator{\id}{Id}
\DeclareMathOperator{\len}{len}

\DeclareMathOperator{\gl}{GL}
\DeclareMathOperator{\aff}{Aff}
\DeclareMathOperator{\isom}{Isom}







\title[Distanza di Hausdorff]{Disatanza di Hausdorff e applicazioni
  in computer vision}
%\subtitle[Unified Zone]{Gestione dei servizi offerti da UZ: Unified Zone}
\author{Enrico Polesel}
\institute[Scuola Normale Superiore]{Scuola Normale Superiore}
\date{20 marzo 2014}


\begin{document}

\begin{frame}[plain]
  \titlepage
\end{frame}

\begin{frame}
 \frametitle{Indice}
 \tableofcontents
\end{frame}


%\AtBeginSection[]
%{
%  \begin{frame}{\secname}
%    \tableofcontents[currentsection]
%  \end{frame}
%}


\AtBeginSubsection[]
{
  \begin{frame}{\secname $\rightarrow$ \subsecname}
    \tableofcontents[currentsubsection]
  \end{frame}
}

\begin{frame}{Distanza da un insieme}
  Consideriamo $(M,d)$ spazio metrico e $\Omega \subseteq M$ spazio
  ``ambiente'' nel quale vogliamo considerare i sottoinsiemi.
  
  \begin{mydef}[Distanza di un punto da un insieme]
    Dato $A \subseteq \Omega$ non vuoto definiamo:
    
    \[  d_A (x) := \inf _{y \in A} d(x,y) \] 
  \end{mydef}
  
  Con questo possiamo definire lo spazio di forme
  
  \[ C_d(\Omega) = \set{d_A : A \subseteq \Omega,\; A \neq \emptyset} \]
\end{frame}

\begin{frame}{Proprietà della distanza da un insieme}
  \begin{mypro}[Proprietà della distanza punto-insieme]
    \begin{itemize}
    \item La funzione $x \rightarrow d_A(x)$ è 1-Lipschitz (e quindi
      continua)
    \item $d_{A\cup B}(x) = \min\set{d_A(x),d_B(x)}$
    \item $\set{x:d_A(x) = 0} = \obar A$
    \item $d_A \equiv 0 \Rightarrow A$ è denso in $\Omega$
    \item $d_{\obar A} = d_A$
    \item $A\subseteq B \Rightarrow d_A \ge d_B$
    \item $\obar{ A} \subseteq \obar B \Leftrightarrow d_A \ge d_B$
    \item $d_A = d_B \Leftrightarrow \obar A = \obar B$
    \end{itemize}
  \end{mypro}
\end{frame}

\begin{frame}{Quando $d_A$ \`e in realt\`a un minimo}
  In generale l'$\inf$ della definizione della distanza $d_A$ non \`e
  un minimo, nemmeno chiedendo che $A$ sia chiuso, per esempio:
  \begin{myes}
    In $L^2(\mathbb{N})$ considero i punti $a_n = ( 0, 0, ..., 1 +
    \frac{1}{n} , 0, ...)$ dove il termine non nullo si trova
    all'$n$-esimo posto. Allora $A = \set{ a_n : n\in \mathbb{N}}$ \`e
    chiuso, ma non esiste $a_n$ tale che $d(0,a_n)$ sia minimo
  \end{myes}

  L'$\inf$ diventa un minimo se:
  \begin{itemize}
  \item $A$ \`e compatto
  \item In $M$ le palle chiuse sono compatte
  \end{itemize}
\end{frame}



\end{document}
